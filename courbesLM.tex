\documentclass{report}
\usepackage{graphics}
\usepackage{subfig}
\usepackage[T1]{fontenc}
\usepackage[utf8]{inputenc}
\usepackage[french]{babel}
\usepackage{amsfonts,amssymb,amsmath,amsthm}
\usepackage{graphicx}

\begin{document}
On propose une autre association courbe discrete - courbe continue que celle proposée par Lachaud avec les normales.

Sur la frontière discrete, on distingue deux type de sommets (voir Fig.~\ref{sfig:tube1}):
\begin{itemize}
\item
  entre deux arêtes alignées ; le sommet est alors le centre d'un carré dont deux sommets adjacents sont allumés et les deux autres sont éteints.
\item
  entre deux arêtes orthogonales ; le sommet est alors le centre de l'hypothénuse d'un triangle rectangle isocèle dont l'angle droit est éteint et les deux autres sommets allumés, ou inversement.
\end{itemize}
  
\begin{figure}[!h]
  \centering
  \subfloat[]{\label{sfig:tube1}\includegraphics[scale=0.65]{tube.pdf}}
  \hspace{1cm}
  \subfloat[)]{\label{sfig:tube3}\includegraphics[scale=0.65,clip=true,trim=0 0 100 40]{tube3.pdf}}
  \caption{\label{fig:tube}(a) En vert : la courbe continue. Points noirs : discrétisation de Gauss (points allumés). Points blancs : points éteints. En rouge : la courbe discrète. En bleu : les carrés et triangles décrits dans le texte.
  (b) Idem avec des \og carrés\fg{} et des \og triangles\fg{} qui tiennent compte de la courbure maximale de la courbe.}
\end{figure}


En prenant comme hypothèse la par($r$)-regularité et $h<r$, on devrait pouvoir montrer que la courbe est tout entière incluse dans l'union de ces carrés et triangles sous reserve de les gonfler un peu comme sur la figure~\ref{fig:carres_triangles}. Le \og tube\fg{} obtenu est affiché sur la figure~\ref{sfig:tube3}.
\begin{figure}[h!]
  \centering
  \includegraphics[width=8cm,clip=true,trim=0 210 0 0]{tube3-1}
    \caption{\label{fig:carres_triangles}Les arcs de cercle ont pour rayon $r$ avec ici $r=h$ soit le minimum puisque, par hypothèse, $h<r$.}
    \end{figure}

    Sous reserve d'arriver à montrer que l'intersection de la courbe avec une structure carré ou triangle est connexe, on associera à chaque sommet de la courbe discrète le centre (relativement à la paramétrisation) du segment de courbe délimité par la structure associée au sommet. Pour l'instant, on nomme $\Pi$ cette association.

    \bigskip
    
    \hfill TSVP $\to$

    \newpage. 
    \paragraph{Parcours d'une structure carré}
    On se propose de montrer que la courbe ne franchit qu'une seule fois la \og porte d'entrée\fg{} d'une structure carré avant de ressortir de l'autre côté.
    Les notations sont celles de la figure~\ref{fig:normale_horiz}.

    \begin{proof}
      Soient $A$, $B$, $C$, $D$, les sommets de la structure carré. Raisonnons par l'absurde. Si la courbe franchit $[AD]$ une première fois puis une seconde fois avant d'avoir franchit la \og porte de sortie\fg{} $[BC]$, par le théorème des valeurs intermédiaires sur la composante horizontale de la dérivée, il existe un point dans la structure carré où la normale est horizontale (parallèle à $[AB]$).
      Soit $E$ un tel point. Les centres des boules tangentes intérieures et extérieures sont alors les points $F$ et $G$. Disons par exemple que $F$ est le centre de la boule tangente intérieure et $G$ le centre de la boule tangente extérieure. Alors, $F$ ne doit pas être dans le disque $d_D$ de centre $D$ et de rayon $h$ sinon la boule tangente intérieure contiendrait le point extérieur $D$. De même, $G$ ne doit pas être dans le disque $d_B$ de centre $B$ et de rayon $h$.
      On en déduit que $E$ ne peut pas être dans le translaté de $d_D$ de vecteur $\vec {FE}$, c'est-à-dire dans le disque $d_C$ de centre $C$ et de rayon 1 ni, symétriquement, dans le disque $d_A$ de centre $A$ et de rayon 1. Comme l'union de ces deux derniers disques recouvrent la structure carré, le point $E$ appartient à l'ensemble vide.
    \end{proof}
    
    \begin{figure}[!h]
      \centering
\includegraphics[width=12cm]{normale_horiz.pdf}
\caption{\label{fig:normale_horiz}Tous les cercles ont pour rayon $h$ et les segments $[AB]$, $[BC]$, $[CD]$, $DA]$, $[EF]$ et $[EG]$ ont pour longueur $h$.}
\end{figure}

La propriété précédente devrait permettre de montrer une propriété de croissance de l'association sommet discret $\to$ point sur courbe : si les sommets discrets $x$, $y$, $z$ sont tels que $y$ est entre $x$ et $z$ alors $\Pi(y)$ est entre $\Pi(x)$ et $\Pi(z)$.

\newpage

Sur la forme des bords des structures\\
En raisonnant sur les disques interdits, on obtient facilement que la zone est un ensemble de cercles passant par un des deux points à éviter et faisant un angle plus ou moins important avec la ligne reliant les deux points.
La configuration limite est celle où l'angle est maximal. Elle correspond au cercle dont le centre est l'intersection des deux cercles du début.

Sur la connexité\\
Si l'intersection de la courbe avec une structure a deux composantes connexes, on a vu que ces composantes traversent de part en part la structure.
Or, les diamètres des boules tangentes est supérieure au diamètre de la structure. Donc, le diamètre de la boule va traverser la deuxième composante, ce qui est interdit.

Sur l'extension aux courbes C0\\
Remplacer la par-regularité par la visibilité d'une paires de points allumés/éteints contigus sous un angle minimal et sans traverser de région.

Sur la relation d'ordre\\
En considérant deux structures contigues et en appliquant le résultat sur l'unicité du franchissement des portes des structures, on obtient le résultat.
\end{document}
